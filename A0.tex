\documentclass{article}
\usepackage[utf8]{inputenc}
% Uncomment if using graphics
% \usepackage{graphicx}
% \graphicspath{{/home/alex/Laptop-Server/Reproduction-Schema-WriteUp/images/}}
\usepackage[margin=1in]{geometry}
\usepackage{cancel}
\usepackage{amsmath}
\usepackage{amsthm}
\usepackage{amssymb}
\usepackage{verbatim}
\usepackage{hyperref}

\hypersetup{
    colorlinks=true,
    linkcolor=blue,
    filecolor=magenta,      
    urlcolor=cyan,
    pdftitle={Overleaf Example},
    pdfpagemode=FullScreen,
}
\urlstyle{same}

\title{Assignment 0}
\author{Yavuz Damkaci}
\date{}

\newtheorem{lemma}{Lemma}[section]

\begin{document}

\maketitle

\begin{enumerate}
    \item By the chain rule from calculus, we have:
    \begin{align}
        \frac{d}{dx}\sin(x^2 + 6x) &= \cos(x^2 + 6x)\frac{d}{dx}(x^2 + 6x) \\
        &= \cos(x^2 + 6x)(2x+6)
    \end{align}
    
    \item DeMorgan's law says
    \begin{equation}
        \neg(A \cap B) \equiv \neg A \cup \neg B
    \end{equation}
    
    \item
    \[
    \begin{array}{|c|c|c|}
        \hline
        A & B & A \land B \\
        \hline
        T & T & T \\
        T & F & F \\
        F & T & F \\
        F & F & F \\
        \hline
    \end{array}
    \]
    
    \item $\mathbb{R}$ denotes the real numbers. $\mathbb{N}$ denotes the natural numbers. It is of course the case that $\mathbb{N} \subseteq \mathbb{R}$
\end{enumerate}

\end{document}

